\documentclass[12pt,preview]{standalone}

\usepackage[utf8x]{inputenc}
\usepackage{amsmath}
\usepackage{amsfonts}
\usepackage{amssymb}
\usepackage{newtxtext}
\usepackage[libertine]{newtxmath}

\newcommand{\FLPvec}[1]{\boldsymbol{#1}}
\newcommand{\Figvec}[1]{\mathbf{#1}}
\newcommand{\FLPC}{\FLPvec{C}}
\newcommand{\FLPF}{\FLPvec{F}}
\newcommand{\FLPa}{\FLPvec{a}}
\newcommand{\FLPb}{\FLPvec{a}}
\newcommand{\FLPr}{\FLPvec{r}}
\newcommand{\FLPs}{\FLPvec{s}}
\newcommand{\FLPv}{\FLPvec{v}}
\newcommand{\ddt}[2]{\frac{d#1}{d#2}}
\newcommand{\epsO}{\epsilon_0}
\newcommand{\FigC}{\Figvec{C}}

\begin{document}
\begin{preview}
    \begin{equation}
        \underset{\text{K.E.}}{\tfrac{1}{2}mv^2}+
        \underset{\text{P.E.}}{\vphantom{\tfrac{1}{2}}mgh}=\text{const},\notag
        \end{equation}
        \begin{equation}
        \label{Eq:I:13:1}
        T+U=\text{const}.
        \end{equation}
        \begin{equation}
        \label{Eq:I:13:2}
        \ddt{T}{t}=\ddt{}{t}\,(\tfrac{1}{2}mv^2)=
        \tfrac{1}{2}m2v\,\ddt{v}{t}=mv\,\ddt{v}{t},
    \end{equation}
\end{preview}
\end{document}